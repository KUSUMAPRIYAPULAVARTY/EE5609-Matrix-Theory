%\documentclass[12pt]{article}
\documentclass[12pt]{scrartcl}
\title{EE5609 Matrix Theory - Assignment 1}
\nonstopmode
%\usepackage[utf-8]{inputenc}
\usepackage{graphicx} % Required for including pictures
\usepackage[figurename=Figure]{caption}
\usepackage{float}    % For tables and other floats
\usepackage{verbatim} % For comments and other
\usepackage{amsmath}  % For math
\usepackage{amssymb}  % For more math
\usepackage{fullpage} % Set margins and place page numbers at bottom center
\usepackage{paralist} % paragraph spacing
\usepackage{listings} % For source code
\usepackage{subfig}   % For subfigures
%\usepackage{physics}  % for simplified dv, and 
\usepackage{enumitem} % useful for itemization
\usepackage{siunitx}  % standardization of si units

\usepackage{tikz,bm} % Useful for drawing plots
%\usepackage{tikz-3dplot}
\usepackage{circuitikz}

%%% Colours used in field vectors and propagation direction
\definecolor{mycolor}{rgb}{1,0.2,0.3}
\definecolor{brightgreen}{rgb}{0.4, 1.0, 0.0}
\definecolor{britishracinggreen}{rgb}{0.0, 0.26, 0.15}
\definecolor{cadmiumgreen}{rgb}{0.0, 0.42, 0.24}
\definecolor{ceruleanblue}{rgb}{0.16, 0.32, 0.75}
\definecolor{darkelectricblue}{rgb}{0.33, 0.41, 0.47}
\definecolor{darkpowderblue}{rgb}{0.0, 0.2, 0.6}
\definecolor{darktangerine}{rgb}{1.0, 0.66, 0.07}
\definecolor{emerald}{rgb}{0.31, 0.78, 0.47}
\definecolor{palatinatepurple}{rgb}{0.41, 0.16, 0.38}
\definecolor{pastelviolet}{rgb}{0.8, 0.6, 0.79}
\begin{document}

\begin{center}
	\hrule
	\vspace{.4cm}
	{\textbf { \large  Matrix Theory - Assignment 1}}
\end{center}
{ \textbf{Name:}} \ Sri Harsha CH \hspace{\fill} \textbf{Roll No:} AI20MTECH14007 \\
	\hrule

\paragraph*{Problem 1:} %\hfill \newline
 Write down a unit vector in the $xy$\nobreakdash-plane, making an angle of $30^{\circ}$ with the positive direction of the $x$-axis ?


\paragraph*{Solution: } %\hfill \newline
 Let us consider a unit vector $\vec{a}$ in the $xy$\nobreakdash-plane, and given this vector makes an angle of $30^{\circ}$ with the positive direction of the $x$-axis.\\
 With the angle given, we can find out the slope $m$ and using this slope we can find the direction vector.\\
 The slope($m$) is given by:\\
 \begin{align*}
& m = \tan{\theta}\\ 
\end{align*}
 and the direction vector is obtained from slope as:\\
 \begin{align*}
& \begin{pmatrix} 1 \\ m \end{pmatrix}
\end{align*}
Substituting $\theta$ = $30^{\circ}$ in slope equation, we get:
 \begin{align*}
& m = \tan{30^{\circ}} = \frac{1}{\sqrt{3}}\\ 
\end{align*}
and the direction vector is:
 \begin{align*}
& \vec{a}  = \begin{pmatrix} 1 \\ \frac{1}{\sqrt{3}} \end{pmatrix}
\end{align*}

To find a unit vector with the same direction as direction vector, we divide by the magnitude of the vector.
 \begin{align*}
& \hat{a}  = \frac{\vec{a}}{\lvert \vec{a} \rvert}\\
& \lvert \vec{a} \rvert  = \sqrt{ \left( 1 \right)^{\!\!2} +  \left( \frac{1}{\sqrt{3}} \right)^{\!\!2}} = \frac{2}{\sqrt{3}}
\end{align*}

$\implies$ The unit vector is given by:
 \begin{align*}
& \hat{a}  = \begin{pmatrix} \frac{1}{\frac{2}{\sqrt{3}}} \\ \frac{\frac{1}{\sqrt{3}}}{\frac{2}{\sqrt{3}}} \end{pmatrix}\\
& \hat{a}  = \begin{pmatrix} \frac{\sqrt{3}}{2} \\ \frac{1}{2} \end{pmatrix}\\
& \implies \boxed{\hat{a}  = \begin{pmatrix} \frac{\sqrt{3}}{2} \\ \frac{1}{2} \end{pmatrix}}\\
\end{align*}
\end{document}
